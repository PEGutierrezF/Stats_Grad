% Options for packages loaded elsewhere
\PassOptionsToPackage{unicode}{hyperref}
\PassOptionsToPackage{hyphens}{url}
%
\documentclass[
  ignorenonframetext,
]{beamer}
\usepackage{pgfpages}
\setbeamertemplate{caption}[numbered]
\setbeamertemplate{caption label separator}{: }
\setbeamercolor{caption name}{fg=normal text.fg}
\beamertemplatenavigationsymbolsempty
% Prevent slide breaks in the middle of a paragraph
\widowpenalties 1 10000
\raggedbottom
\setbeamertemplate{part page}{
  \centering
  \begin{beamercolorbox}[sep=16pt,center]{part title}
    \usebeamerfont{part title}\insertpart\par
  \end{beamercolorbox}
}
\setbeamertemplate{section page}{
  \centering
  \begin{beamercolorbox}[sep=12pt,center]{part title}
    \usebeamerfont{section title}\insertsection\par
  \end{beamercolorbox}
}
\setbeamertemplate{subsection page}{
  \centering
  \begin{beamercolorbox}[sep=8pt,center]{part title}
    \usebeamerfont{subsection title}\insertsubsection\par
  \end{beamercolorbox}
}
\AtBeginPart{
  \frame{\partpage}
}
\AtBeginSection{
  \ifbibliography
  \else
    \frame{\sectionpage}
  \fi
}
\AtBeginSubsection{
  \frame{\subsectionpage}
}
\usepackage{amsmath,amssymb}
\usepackage{iftex}
\ifPDFTeX
  \usepackage[T1]{fontenc}
  \usepackage[utf8]{inputenc}
  \usepackage{textcomp} % provide euro and other symbols
\else % if luatex or xetex
  \usepackage{unicode-math} % this also loads fontspec
  \defaultfontfeatures{Scale=MatchLowercase}
  \defaultfontfeatures[\rmfamily]{Ligatures=TeX,Scale=1}
\fi
\usepackage{lmodern}
\usetheme[]{Boadilla}
\ifPDFTeX\else
  % xetex/luatex font selection
\fi
% Use upquote if available, for straight quotes in verbatim environments
\IfFileExists{upquote.sty}{\usepackage{upquote}}{}
\IfFileExists{microtype.sty}{% use microtype if available
  \usepackage[]{microtype}
  \UseMicrotypeSet[protrusion]{basicmath} % disable protrusion for tt fonts
}{}
\makeatletter
\@ifundefined{KOMAClassName}{% if non-KOMA class
  \IfFileExists{parskip.sty}{%
    \usepackage{parskip}
  }{% else
    \setlength{\parindent}{0pt}
    \setlength{\parskip}{6pt plus 2pt minus 1pt}}
}{% if KOMA class
  \KOMAoptions{parskip=half}}
\makeatother
\usepackage{xcolor}
\newif\ifbibliography
\usepackage{color}
\usepackage{fancyvrb}
\newcommand{\VerbBar}{|}
\newcommand{\VERB}{\Verb[commandchars=\\\{\}]}
\DefineVerbatimEnvironment{Highlighting}{Verbatim}{commandchars=\\\{\}}
% Add ',fontsize=\small' for more characters per line
\usepackage{framed}
\definecolor{shadecolor}{RGB}{248,248,248}
\newenvironment{Shaded}{\begin{snugshade}}{\end{snugshade}}
\newcommand{\AlertTok}[1]{\textcolor[rgb]{0.94,0.16,0.16}{#1}}
\newcommand{\AnnotationTok}[1]{\textcolor[rgb]{0.56,0.35,0.01}{\textbf{\textit{#1}}}}
\newcommand{\AttributeTok}[1]{\textcolor[rgb]{0.13,0.29,0.53}{#1}}
\newcommand{\BaseNTok}[1]{\textcolor[rgb]{0.00,0.00,0.81}{#1}}
\newcommand{\BuiltInTok}[1]{#1}
\newcommand{\CharTok}[1]{\textcolor[rgb]{0.31,0.60,0.02}{#1}}
\newcommand{\CommentTok}[1]{\textcolor[rgb]{0.56,0.35,0.01}{\textit{#1}}}
\newcommand{\CommentVarTok}[1]{\textcolor[rgb]{0.56,0.35,0.01}{\textbf{\textit{#1}}}}
\newcommand{\ConstantTok}[1]{\textcolor[rgb]{0.56,0.35,0.01}{#1}}
\newcommand{\ControlFlowTok}[1]{\textcolor[rgb]{0.13,0.29,0.53}{\textbf{#1}}}
\newcommand{\DataTypeTok}[1]{\textcolor[rgb]{0.13,0.29,0.53}{#1}}
\newcommand{\DecValTok}[1]{\textcolor[rgb]{0.00,0.00,0.81}{#1}}
\newcommand{\DocumentationTok}[1]{\textcolor[rgb]{0.56,0.35,0.01}{\textbf{\textit{#1}}}}
\newcommand{\ErrorTok}[1]{\textcolor[rgb]{0.64,0.00,0.00}{\textbf{#1}}}
\newcommand{\ExtensionTok}[1]{#1}
\newcommand{\FloatTok}[1]{\textcolor[rgb]{0.00,0.00,0.81}{#1}}
\newcommand{\FunctionTok}[1]{\textcolor[rgb]{0.13,0.29,0.53}{\textbf{#1}}}
\newcommand{\ImportTok}[1]{#1}
\newcommand{\InformationTok}[1]{\textcolor[rgb]{0.56,0.35,0.01}{\textbf{\textit{#1}}}}
\newcommand{\KeywordTok}[1]{\textcolor[rgb]{0.13,0.29,0.53}{\textbf{#1}}}
\newcommand{\NormalTok}[1]{#1}
\newcommand{\OperatorTok}[1]{\textcolor[rgb]{0.81,0.36,0.00}{\textbf{#1}}}
\newcommand{\OtherTok}[1]{\textcolor[rgb]{0.56,0.35,0.01}{#1}}
\newcommand{\PreprocessorTok}[1]{\textcolor[rgb]{0.56,0.35,0.01}{\textit{#1}}}
\newcommand{\RegionMarkerTok}[1]{#1}
\newcommand{\SpecialCharTok}[1]{\textcolor[rgb]{0.81,0.36,0.00}{\textbf{#1}}}
\newcommand{\SpecialStringTok}[1]{\textcolor[rgb]{0.31,0.60,0.02}{#1}}
\newcommand{\StringTok}[1]{\textcolor[rgb]{0.31,0.60,0.02}{#1}}
\newcommand{\VariableTok}[1]{\textcolor[rgb]{0.00,0.00,0.00}{#1}}
\newcommand{\VerbatimStringTok}[1]{\textcolor[rgb]{0.31,0.60,0.02}{#1}}
\newcommand{\WarningTok}[1]{\textcolor[rgb]{0.56,0.35,0.01}{\textbf{\textit{#1}}}}
\usepackage{longtable,booktabs,array}
\usepackage{calc} % for calculating minipage widths
\usepackage{caption}
% Make caption package work with longtable
\makeatletter
\def\fnum@table{\tablename~\thetable}
\makeatother
\usepackage{graphicx}
\makeatletter
\def\maxwidth{\ifdim\Gin@nat@width>\linewidth\linewidth\else\Gin@nat@width\fi}
\def\maxheight{\ifdim\Gin@nat@height>\textheight\textheight\else\Gin@nat@height\fi}
\makeatother
% Scale images if necessary, so that they will not overflow the page
% margins by default, and it is still possible to overwrite the defaults
% using explicit options in \includegraphics[width, height, ...]{}
\setkeys{Gin}{width=\maxwidth,height=\maxheight,keepaspectratio}
% Set default figure placement to htbp
\makeatletter
\def\fps@figure{htbp}
\makeatother
\setlength{\emergencystretch}{3em} % prevent overfull lines
\providecommand{\tightlist}{%
  \setlength{\itemsep}{0pt}\setlength{\parskip}{0pt}}
\setcounter{secnumdepth}{-\maxdimen} % remove section numbering
\usepackage{graphicx}
\logo{\ifnum\thepage>1\hfill\includegraphics[width=1cm]{logo}\fi}
\titlegraphic{\includegraphics[width=3cm]{logo}}
\newcommand{\theHtable}{\thetable}
\ifLuaTeX
  \usepackage{selnolig}  % disable illegal ligatures
\fi
\usepackage{bookmark}
\IfFileExists{xurl.sty}{\usepackage{xurl}}{} % add URL line breaks if available
\urlstyle{same}
\hypersetup{
  pdftitle={Hypothesis Testing, Probability and Distributions},
  pdfauthor={Pablo E. Gutierrez-Fonseca},
  hidelinks,
  pdfcreator={LaTeX via pandoc}}

\title{Hypothesis Testing, Probability and Distributions}
\author{Pablo E. Gutierrez-Fonseca}
\date{Fall 2024}

\begin{document}
\frame{\titlepage}

\begin{frame}{Normal distribution: Introduction}
\phantomsection\label{normal-distribution-introduction}
\begin{itemize}
\tightlist
\item
  Why did we focus on normality?
\end{itemize}

\begin{itemize}
\tightlist
\item
  The \textbf{normal distribution} is a key tool for determining the
  probability of a given value occurring in a population that follows
  this distribution.
\end{itemize}

\begin{itemize}
\tightlist
\item
  It allows us to make inferences about a population by calculating how
  likely it is for data to fall within certain ranges.
\end{itemize}

\begin{itemize}
\tightlist
\item
  Many statistical tests assume data follows a \textbf{normal
  distribution}, which helps in determining \textbf{significance} and
  making reliable conclusions.
\end{itemize}
\end{frame}

\begin{frame}{Normal distribution: Introduction}
\phantomsection\label{normal-distribution-introduction-1}
\begin{itemize}
\tightlist
\item
  The normal distribution has two parameters: the mean (\(\mu\)) and the
  standard deviation (\(\sigma\)).
\end{itemize}

\begin{itemize}
\tightlist
\item
  If a \textbf{variable x} follows a normal distribution with a specific
  mean (\(\mu\)) and standard deviation (\(\sigma\)), we write this as:
\end{itemize}

\begin{columns}[T]
\begin{column}{0.48\textwidth}
\includegraphics{M5-Hypothesis-Testing,-Probability-and-Distribution_files/figure-beamer/unnamed-chunk-1-1.pdf}
\end{column}

\begin{column}{0.48\textwidth}
\vspace{2cm}

\[
x \sim N(\mu, \sigma)
\]
\end{column}
\end{columns}
\end{frame}

\begin{frame}{Converting to Standard Normal Distribution}
\phantomsection\label{converting-to-standard-normal-distribution}
\begin{itemize}
\tightlist
\item
  To simplify comparisons between different normal distributions, we
  often convert values to the \textbf{standard normal distribution}.
\end{itemize}

\begin{itemize}
\tightlist
\item
  The standard normal distribution has a \(\mu\) = \(0\) and a
  \(\sigma\) = \(1\).
\end{itemize}

\begin{itemize}
\tightlist
\item
  Any normal distribution can be converted into the standard normal
  distribution by using the \textbf{z-score} formula:
  \(z = \frac{x - \mu}{\sigma}\)
\end{itemize}

\includegraphics{M5-Hypothesis-Testing,-Probability-and-Distribution_files/figure-beamer/unnamed-chunk-2-1.pdf}
\end{frame}

\begin{frame}{Hypothesis Testing, Probability and Distributions}
\phantomsection\label{hypothesis-testing-probability-and-distributions}
\begin{itemize}
\tightlist
\item
  Hypothesis Testing

  \begin{itemize}
  \tightlist
  \item
    Null and alternative hypothesis
  \item
    Test statistic
  \end{itemize}
\end{itemize}

\begin{itemize}
\tightlist
\item
  \textcolor{gray}{Probability and Distributions}
\end{itemize}

\begin{itemize}
\tightlist
\item
  \textcolor{gray}{z-score}
\end{itemize}
\end{frame}

\begin{frame}{Hypothesis Testing: Null and alternative hypothesis}
\phantomsection\label{hypothesis-testing-null-and-alternative-hypothesis}
\begin{itemize}
\tightlist
\item
  A \textbf{null hypothesis} (denoted by \(H_0\)) is a statement of the
  status quo, one of \textbf{no difference or no effect}.

  \begin{itemize}
  \tightlist
  \item
    If the null hypothesis is not rejected, no changes will be made.
  \end{itemize}
\end{itemize}

\centering

\(H_0\): \(\mu = \mu_0\)\\

\vspace{1cm}

\begin{itemize}
\item
  An \textbf{alternative hypothesis} (denoted by \(H_1\)) is one in
  which \textbf{some difference or effect is expected}.
\item
  One sided alternative hypothesis:
\end{itemize}

\centering

\(H_1\): \(\mu < \mu_0\) or \(H_1\): \(\mu > \mu_0\)\\

\begin{itemize}
\tightlist
\item
  Two sided alternative hypothesis:
\end{itemize}

\centering

\(H_1\): \(\mu \neq \mu_0\)\\
\end{frame}

\begin{frame}{Hypothesis Testing: Test statistic}
\phantomsection\label{hypothesis-testing-test-statistic}
\begin{itemize}
\tightlist
\item
  All inferential tests use a formula that calculates a \textbf{test
  statistic}, quantifying the relationship or difference you are
  testing.
\end{itemize}

\begin{columns}[T]
\begin{column}{0.48\textwidth}
\small

\begin{itemize}
\item
  Independent t-test:
  \[t = \frac{\bar{X_1} - \bar{X_2}}{\sqrt{\frac{s_1^2}{n_1} + \frac{s_2^2}{n_2}}}\]

  \begin{itemize}
  \item
    Dependent t-test \[t = \frac{\bar{D}}{\frac{s_D}{\sqrt{n}}}\]
  \item
    One sample z-test
    \[z = \frac{\bar{X} - \mu}{\frac{\sigma}{\sqrt{n}}}\] \small  
  \end{itemize}
\end{itemize}
\end{column}

\begin{column}{0.48\textwidth}
\small

\begin{itemize}
\item
  F-test (ANOVA) \[F = \frac{MST}{MSE}\]
\item
  Pearson correlation
  \[r = \frac{\sum (x_i - \bar{x})(y_i - \bar{y})}{\sqrt{\sum (x_i - \bar{x})^2 \sum (y_i - \bar{y})^2}}\]\\
\end{itemize}
\end{column}
\end{columns}
\end{frame}

\begin{frame}{Hypothesis Testing: Test statistic}
\phantomsection\label{hypothesis-testing-test-statistic-1}
\begin{itemize}
\tightlist
\item
  The \textbf{normal distribution curve} allows us to calculate the
  \textbf{probability} of obtaining a \textbf{test statistic} as extreme
  as the one observed, assuming the \textbf{Null Hypothesis is true}.
\end{itemize}

\begin{itemize}
\tightlist
\item
  The \textbf{test statistic} is a value used in making a decision about
  the null hypothesis, and is found by converting the sample statistic
  to a score with the assumption that the null hypothesis is true
\end{itemize}

\includegraphics{M5-Hypothesis-Testing,-Probability-and-Distribution_files/figure-beamer/unnamed-chunk-8-1.pdf}
\end{frame}

\begin{frame}{Hypothesis Testing: Test statistic}
\phantomsection\label{hypothesis-testing-test-statistic-2}
\begin{itemize}
\tightlist
\item
  If the \textbf{probability} (i.e., \emph{the area under the curve}) of
  obtaining such an extreme test statistic is less than our chosen
  significance level (\emph{usually 0.05}), we consider the result to be
  \textbf{statistically significant}.
\end{itemize}

\begin{itemize}
\tightlist
\item
  In such cases, we may \textbf{reject the Null Hypothesis}, suggesting
  there is evidence for a difference or effect.
  \includegraphics{M5-Hypothesis-Testing,-Probability-and-Distribution_files/figure-beamer/unnamed-chunk-9-1.pdf}
\end{itemize}
\end{frame}

\begin{frame}{Hypothesis Testing: Test statistic}
\phantomsection\label{hypothesis-testing-test-statistic-3}
\begin{itemize}
\tightlist
\item
  Another way to think about this:
\end{itemize}

\begin{columns}[T]
\begin{column}{0.48\textwidth}
\small

\begin{itemize}
\tightlist
\item
  If the \textbf{probability that our data fits the null distribution
  (i.e., the null hypothesis is true) is less than 5\%}, we conclude
  that the data \textbf{does not fit the null}.
\item
  This indicates a significant deviation from what we would expect by
  chance.
\item
  We then \textbf{reject the null hypothesis}.
\item
  The result is considered \textbf{statistically significant}.\\
\end{itemize}
\end{column}

\begin{column}{0.48\textwidth}
\begin{figure}

{\centering \includegraphics[width=0.8\linewidth]{figs/plot1} 

}

\end{figure}
\end{column}
\end{columns}
\end{frame}

\begin{frame}{Hypothesis Testing, Probability and Distributions}
\phantomsection\label{hypothesis-testing-probability-and-distributions-1}
\begin{itemize}
\tightlist
\item
  \textcolor{gray}{Hypothesis Testing}
\end{itemize}

\begin{itemize}
\tightlist
\item
  Probability and Distributions
\end{itemize}

\begin{itemize}
\tightlist
\item
  \textcolor{gray}{z-score}
\end{itemize}
\end{frame}

\begin{frame}{Before: p-value, significance and alfa level}
\phantomsection\label{before-p-value-significance-and-alfa-level}
\end{frame}

\begin{frame}{Why p-value of less than \textbf{0.05}?}
\phantomsection\label{why-p-value-of-less-than-0.05}
\begin{columns}[T]
\begin{column}{0.48\textwidth}
``It is usual and convenient for experimenters to take 5\% as a standard
level of significance, in the sense that they are prepared to ignore all
results which fail to reach this standard, and, by this means, to
eliminate from further discussion the greater part of the fluctuations
which chance causes have introduced into their experimental results.''
\end{column}

\begin{column}{0.48\textwidth}
\begin{itemize}
\tightlist
\item
  Ronald Aylmer Fisher (1890-1962)
\end{itemize}

\begin{figure}
\includegraphics[width=0.8\linewidth]{figs/fisher} \end{figure}
\end{column}
\end{columns}
\end{frame}

\begin{frame}{What is a P-Value?}
\phantomsection\label{what-is-a-p-value}
\begin{itemize}
\tightlist
\item
  The \textbf{p-value} (or p-value or probability value) is the
  probability of getting a value of the \textbf{test statistic} that is
  \textbf{at least as extreme} as the one representing the sample data,
  assuming that the null hypothesis is true
\end{itemize}

\begin{itemize}
\item
  Interpretation:

  \begin{itemize}
  \tightlist
  \item
    A small p-value indicates that the observed result is unlikely under
    the null hypothesis, suggesting evidence against it.\\
  \item
    A large p-value suggests that the observed result is consistent with
    the null hypothesis.
  \end{itemize}
\end{itemize}
\end{frame}

\begin{frame}{Critical Region}
\phantomsection\label{critical-region}
\begin{itemize}
\tightlist
\item
  The critical region (or rejection region) is the set of all values of
  the test statistic that cause us to reject the null hypothesis.
\end{itemize}

\begin{figure}

{\centering \includegraphics[width=0.4\linewidth]{figs/criticalregion} 

}

\end{figure}

\begin{itemize}
\tightlist
\item
  Acceptance and rejection regions in case of a two-tailed test with 5\%
  significance level.
\end{itemize}
\end{frame}

\begin{frame}{What is Significance?}
\phantomsection\label{what-is-significance}
\begin{columns}[T]
\begin{column}{0.5\textwidth}
\begin{itemize}
\tightlist
\item
  A statistical result is \textbf{significant} if it is \textbf{unlikely
  to have occurred by chance}.
\end{itemize}

\begin{itemize}
\tightlist
\item
  Despite natural variability in the population, the probability of
  observing a value this extreme due to random variability is
  \textbf{low} (though not impossible).
\end{itemize}

\begin{itemize}
\tightlist
\item
  We use probabilities (\textbf{p-values}) and an \textbf{alpha
  threshold (commonly 0.05)} to determine whether a result is
  significant.
\end{itemize}
\end{column}

\begin{column}{0.5\textwidth}
\vspace{1.5cm}
\begin{figure}

{\centering \includegraphics[width=1\linewidth]{figs/plot2} 

}

\end{figure}
\end{column}
\end{columns}
\end{frame}

\begin{frame}{What is Significance?}
\phantomsection\label{what-is-significance-1}
\begin{columns}[T]
\begin{column}{0.5\textwidth}
\begin{itemize}
\tightlist
\item
  Significance refers to the risk of rejecting the null hypothesis when
  it is actually true.
\end{itemize}

\begin{itemize}
\tightlist
\item
  It tells us the probability that our result happened by chance alone.
\end{itemize}

\begin{itemize}
\tightlist
\item
  A p-value of 0.05 (5\%) means there's a 5\% chance the result is due
  to random chance.
\end{itemize}

\begin{itemize}
\tightlist
\item
  A p-value of 0.01 (1\%) means there's a 1\% chance the result happened
  by chance. \tiny

  \begin{itemize}
  \tightlist
  \item
    A p-value of 0.01 indicates a low chance (1\%) of the result
    occurring by chance, reflecting a more rigorous threshold for
    significance.\\
  \end{itemize}
\end{itemize}
\end{column}

\begin{column}{0.5\textwidth}
\vspace{1.5cm}
\begin{figure}

{\centering \includegraphics[width=0.7\linewidth]{figs/significance} 

}

\end{figure}
\end{column}
\end{columns}
\end{frame}

\begin{frame}{Focus - Types of Error}
\phantomsection\label{focus---types-of-error}
\end{frame}

\begin{frame}{Focus - Types of Error}
\phantomsection\label{focus---types-of-error-1}
\begin{longtable}[]{@{}
  >{\raggedright\arraybackslash}p{(\columnwidth - 4\tabcolsep) * \real{0.1890}}
  >{\raggedright\arraybackslash}p{(\columnwidth - 4\tabcolsep) * \real{0.3963}}
  >{\raggedright\arraybackslash}p{(\columnwidth - 4\tabcolsep) * \real{0.4146}}@{}}
\toprule\noalign{}
\begin{minipage}[b]{\linewidth}\raggedright
Scenario
\end{minipage} & \begin{minipage}[b]{\linewidth}\raggedright
Null Hypothesis is True
\end{minipage} & \begin{minipage}[b]{\linewidth}\raggedright
Null Hypothesis is False
\end{minipage} \\
\midrule\noalign{}
\endhead
\textbf{Reject Null Hypothesis} & \textbf{Type I Error}: Incorrectly
rejecting the null hypothesis. & \textbf{Correct Decision}: Correctly
rejecting the null hypothesis. \\
\textbf{Fail to Reject Null Hypothesis} & \textbf{Correct Decision}:
Correctly not rejecting the null hypothesis. & \textbf{Type II Error}:
Incorrectly failing to reject the null hypothesis. \\
\bottomrule\noalign{}
\end{longtable}
\end{frame}

\begin{frame}{Focus - Types of Error}
\phantomsection\label{focus---types-of-error-2}
\begin{longtable}[]{@{}
  >{\raggedright\arraybackslash}p{(\columnwidth - 4\tabcolsep) * \real{0.1890}}
  >{\raggedright\arraybackslash}p{(\columnwidth - 4\tabcolsep) * \real{0.3963}}
  >{\raggedright\arraybackslash}p{(\columnwidth - 4\tabcolsep) * \real{0.4146}}@{}}
\toprule\noalign{}
\begin{minipage}[b]{\linewidth}\raggedright
Scenario
\end{minipage} & \begin{minipage}[b]{\linewidth}\raggedright
Null Hypothesis is True
\end{minipage} & \begin{minipage}[b]{\linewidth}\raggedright
Null Hypothesis is False
\end{minipage} \\
\midrule\noalign{}
\endhead
\textbf{Reject Null Hypothesis} & \textbf{Type I Error}: Incorrectly
rejecting the null hypothesis. & \textbf{Correct Decision}: Correctly
rejecting the null hypothesis. \\
\textbf{Fail to Reject Null Hypothesis} & \textbf{Correct Decision}:
Correctly not rejecting the null hypothesis. & \textbf{Type II Error}:
Incorrectly failing to reject the null hypothesis. \\
\bottomrule\noalign{}
\end{longtable}

\small - Failing to reject the null hypothesis means our data didn't
show a significant effect. It doesn't prove the null hypothesis is true;
it just means we didn't find strong evidence against it.\\
\end{frame}

\begin{frame}{}
\phantomsection\label{section}
\begin{itemize}
\item
  \textbf{Explanation}:

  \begin{itemize}
  \tightlist
  \item
    \textbf{Type I Error}: False positive. We conclude there is an
    effect or difference when there is none.\\
  \item
    \textbf{Type II Error}: False negative. We fail to detect an effect
    or difference when one exists.\\
  \item
    \textbf{Correct Decisions}: Accurately concluding the presence or
    absence of an effect or difference based on the truth of the null
    hypothesis.
  \end{itemize}
\end{itemize}
\end{frame}

\begin{frame}{Calculating Significance}
\phantomsection\label{calculating-significance}
\begin{itemize}
\tightlist
\item
  To quantify a probability, you first need to calculate a \textbf{test
  statistic} and locate it on the normal probability curve.
\end{itemize}

\begin{itemize}
\tightlist
\item
  The normal curve acts as a statistical translator.

  \begin{itemize}
  \tightlist
  \item
    It helps you standardize your test statistic to a common scale.
  \item
    This standardized value is then used to determine the probability of
    obtaining such a result in a standard normal population.
  \end{itemize}
\end{itemize}
\end{frame}

\begin{frame}{Hypothesis Testing, Probability and Distributions}
\phantomsection\label{hypothesis-testing-probability-and-distributions-2}
\begin{itemize}
\tightlist
\item
  \textcolor{gray}{Hypothesis Testing}
\end{itemize}

\begin{itemize}
\tightlist
\item
  \textcolor{gray}{Probability and Distributions}
\end{itemize}

\begin{itemize}
\tightlist
\item
  z-score
\end{itemize}
\end{frame}

\begin{frame}{z-score}
\phantomsection\label{z-score}
\begin{itemize}
\tightlist
\item
  \textbf{z-scores} link measured or hypothesized values to
  probabilities.
\end{itemize}

\begin{itemize}
\tightlist
\item
  A z-score (or standard score) indicates how many standard deviations a
  value \emph{\emph{x}} is above or below the mean on the normal curve.
\end{itemize}

\begin{itemize}
\tightlist
\item
  It helps standardize values and connect them to probabilities.
\end{itemize}

\begin{itemize}
\tightlist
\item
  The z-score is calculated using the formula:
\end{itemize}

\[ z = \frac{(x - \mu)}{\sigma} \]

\begin{itemize}
\tightlist
\item
  where:

  \begin{itemize}
  \tightlist
  \item
    \(x\) = the value of interest.\\
  \item
    \(\mu\) = mean of the population.\\
  \item
    \(\sigma\) = standard deviation of the population.
  \end{itemize}
\end{itemize}
\end{frame}

\begin{frame}{z-scores: Linking observations to probabilities}
\phantomsection\label{z-scores-linking-observations-to-probabilities}
\begin{itemize}
\tightlist
\item
  \textbf{z-scores} link observations to probabilities.
\end{itemize}

\begin{itemize}
\tightlist
\item
  Using a z-score for probability:

  \begin{itemize}
  \tightlist
  \item
    z-scores are essentially the \textbf{x-axis} of the standard normal
    distribution.
  \item
    They normalize any data set so that the mean is 0 and the standard
    deviation is 1.
  \item
    The area under the curve tells you the probability of a certain
    Z-score occurring.
  \item
    By using the Z-score, we can determine the probability associated
    with different values.
  \end{itemize}
\end{itemize}
\end{frame}

\begin{frame}{Finding probabilities for z-scores}
\phantomsection\label{finding-probabilities-for-z-scores}
\begin{columns}[T]
\begin{column}{0.48\textwidth}
\vspace{1cm}

\begin{itemize}
\tightlist
\item
  \textbf{P(X \textless{} z)}: Denotes the probability of a value
  falling \textbf{less than} a given Z-score (\(z\)).
\end{itemize}

\vspace{1cm}

\begin{itemize}
\tightlist
\item
  \textbf{P(X \textgreater{} z)}: Denotes the probability of a value
  falling \textbf{above} a given Z-score (\(z\)).
\end{itemize}

\vspace{1cm}

\begin{itemize}
\tightlist
\item
  \textbf{P(z1 \textless{} X \textless{} z2)}: Denotes the probability
  of a value falling \textbf{between} two different Z-scores (\(z1\) and
  \(z2\)).
\end{itemize}
\end{column}

\begin{column}{0.48\textwidth}
\includegraphics{M5-Hypothesis-Testing,-Probability-and-Distribution_files/figure-beamer/unnamed-chunk-15-1.pdf}
\end{column}
\end{columns}
\end{frame}

\begin{frame}{Example}
\phantomsection\label{example}
\end{frame}

\begin{frame}{Example}
\phantomsection\label{example-1}
\begin{columns}[T]
\begin{column}{0.5\textwidth}
\vspace{1cm}

\begin{itemize}
\tightlist
\item
  What percent of the area under the curve \textbf{falls below} a
  z-score of 0.76?
\end{itemize}

\begin{itemize}
\tightlist
\item
  P (z\textless0.76)
\end{itemize}
\end{column}

\begin{column}{0.5\textwidth}
\vspace{1cm}

\includegraphics{M5-Hypothesis-Testing,-Probability-and-Distribution_files/figure-beamer/unnamed-chunk-16-1.pdf}
\end{column}
\end{columns}
\end{frame}

\begin{frame}[fragile]{Example}
\phantomsection\label{example-2}
\begin{columns}[T]
\begin{column}{0.5\textwidth}
\vspace{1cm}

\begin{itemize}
\tightlist
\item
  What percent of the area under the curve \textbf{falls below} a
  z-score of 0.76?
\end{itemize}

\begin{itemize}
\tightlist
\item
  P (z\textless0.76)
\end{itemize}

\begin{Shaded}
\begin{Highlighting}[]
\CommentTok{\# Calculate the cumulative }
\CommentTok{\# probability for z = 0.76}
\NormalTok{z\_score }\OtherTok{\textless{}{-}} \FloatTok{0.76}
\NormalTok{p\_value }\OtherTok{\textless{}{-}} \FunctionTok{pnorm}\NormalTok{(z\_score)}
\CommentTok{\# Convert to percentage}
\NormalTok{percent\_area }\OtherTok{\textless{}{-}}\NormalTok{ p\_value }\SpecialCharTok{*} \DecValTok{100}
\NormalTok{percent\_area}
\end{Highlighting}
\end{Shaded}

\begin{verbatim}
## [1] 77.63727
\end{verbatim}
\end{column}

\begin{column}{0.5\textwidth}
\vspace{1cm}

\includegraphics{M5-Hypothesis-Testing,-Probability-and-Distribution_files/figure-beamer/unnamed-chunk-18-1.pdf}
\end{column}
\end{columns}
\end{frame}

\begin{frame}{Example}
\phantomsection\label{example-3}
\begin{columns}[T]
\begin{column}{0.5\textwidth}
\vspace{1cm}

\begin{itemize}
\tightlist
\item
  What is the z-score beyond which only 5\% of all possible outcomes are
  higher?
\end{itemize}

\begin{itemize}
\tightlist
\item
  (P\textgreater z) = 0.05
\end{itemize}
\end{column}

\begin{column}{0.5\textwidth}
\vspace{1cm}

\includegraphics{M5-Hypothesis-Testing,-Probability-and-Distribution_files/figure-beamer/unnamed-chunk-19-1.pdf}
\end{column}
\end{columns}
\end{frame}

\begin{frame}[fragile]{Example}
\phantomsection\label{example-4}
\begin{columns}[T]
\begin{column}{0.5\textwidth}
\vspace{1cm}

\begin{itemize}
\tightlist
\item
  What is the z-score beyond which only 5\% of all possible outcomes are
  higher?
\end{itemize}

\begin{itemize}
\tightlist
\item
  (P\textgreater z) = 0.05
\end{itemize}

\begin{Shaded}
\begin{Highlighting}[]
\CommentTok{\# Calculate the z{-}score for }
\CommentTok{\# the upper 5\% (0.95 cumulative }
\CommentTok{\# probability)}
\NormalTok{p\_value }\OtherTok{\textless{}{-}} \FloatTok{0.95}
\NormalTok{z\_score }\OtherTok{\textless{}{-}} \FunctionTok{qnorm}\NormalTok{(p\_value)}
\CommentTok{\# Print the result}
\NormalTok{z\_score}
\end{Highlighting}
\end{Shaded}

\begin{verbatim}
## [1] 1.644854
\end{verbatim}
\end{column}

\begin{column}{0.5\textwidth}
\vspace{1cm}

\includegraphics{M5-Hypothesis-Testing,-Probability-and-Distribution_files/figure-beamer/unnamed-chunk-21-1.pdf}
\end{column}
\end{columns}
\end{frame}

\begin{frame}[fragile]{Example}
\phantomsection\label{example-5}
\begin{columns}[T]
\begin{column}{0.5\textwidth}
\begin{itemize}
\tightlist
\item
  In a distribution with a mean of 100 and a standard deviation of 15,
  what is the probability that a score will fall between 100 and 115 ?
\end{itemize}

\[ Z = \frac{(x - \mu)}{\sigma} \]

\vspace{1cm}

\begin{itemize}
\tightlist
\item
  Before you can find any probabilities you have to find z-scores

  \begin{itemize}
  \tightlist
  \item
    Z = (100 - 100) / 15 = 0
  \item
    Z = (115 - 100) / 15 = 1
  \end{itemize}
\end{itemize}
\end{column}

\begin{column}{0.5\textwidth}
\begin{verbatim}
## [1] "Probability of falling between 100 and 115: 0.3413"
\end{verbatim}

\includegraphics{M5-Hypothesis-Testing,-Probability-and-Distribution_files/figure-beamer/unnamed-chunk-22-1.pdf}
\end{column}
\end{columns}
\end{frame}

\begin{frame}[fragile]{Example}
\phantomsection\label{example-6}
\begin{columns}[T]
\begin{column}{0.5\textwidth}
\begin{itemize}
\tightlist
\item
  In a distribution with a mean of 100 and a standard deviation of 15,
  what is the probability that a score will fall between 100 and 115 ?
\end{itemize}

\begin{Shaded}
\begin{Highlighting}[]
\CommentTok{\# Parameters}
\NormalTok{mean }\OtherTok{\textless{}{-}} \DecValTok{100}
\NormalTok{sd }\OtherTok{\textless{}{-}} \DecValTok{15}
\CommentTok{\# Values}
\NormalTok{low\_value }\OtherTok{\textless{}{-}} \DecValTok{100}
\NormalTok{high\_value }\OtherTok{\textless{}{-}} \DecValTok{115}

\CommentTok{\# Calculate the z{-}scores}
\NormalTok{z\_low }\OtherTok{\textless{}{-}}\NormalTok{ (low\_value}\SpecialCharTok{{-}}\NormalTok{mean)}\SpecialCharTok{/}\NormalTok{sd}
\NormalTok{z\_high }\OtherTok{\textless{}{-}}\NormalTok{ (high\_value}\SpecialCharTok{{-}}\NormalTok{mean)}\SpecialCharTok{/}\NormalTok{sd}
\end{Highlighting}
\end{Shaded}
\end{column}

\begin{column}{0.5\textwidth}
\begin{verbatim}
## [1] "Probability of falling between 100 and 115: 0.3413"
\end{verbatim}

\includegraphics{M5-Hypothesis-Testing,-Probability-and-Distribution_files/figure-beamer/unnamed-chunk-24-1.pdf}
\end{column}
\end{columns}
\end{frame}

\begin{frame}[fragile]{Example}
\phantomsection\label{example-7}
\begin{columns}[T]
\begin{column}{0.5\textwidth}
\begin{itemize}
\tightlist
\item
  In a distribution with a mean of 100 and a standard deviation of 15,
  what is the probability that a score will fall between 100 and 115 ?
\end{itemize}

\begin{Shaded}
\begin{Highlighting}[]
\CommentTok{\# Find the probabilities}
\NormalTok{p\_low }\OtherTok{\textless{}{-}} \FunctionTok{pnorm}\NormalTok{(z\_low)}
\NormalTok{p\_high }\OtherTok{\textless{}{-}} \FunctionTok{pnorm}\NormalTok{(z\_high)}

\CommentTok{\# Probability of falling }
\CommentTok{\# between 100 and 115}
\NormalTok{probability }\OtherTok{\textless{}{-}}\NormalTok{ p\_high }\SpecialCharTok{{-}}\NormalTok{ p\_low}
\NormalTok{probability}
\end{Highlighting}
\end{Shaded}

\begin{verbatim}
## [1] 0.3413447
\end{verbatim}
\end{column}

\begin{column}{0.5\textwidth}
\begin{verbatim}
## [1] "Probability of falling between 100 and 115: 0.3413"
\end{verbatim}

\includegraphics{M5-Hypothesis-Testing,-Probability-and-Distribution_files/figure-beamer/unnamed-chunk-26-1.pdf}
\end{column}
\end{columns}
\end{frame}

\begin{frame}{Example}
\phantomsection\label{example-8}
\end{frame}

\begin{frame}{Example: Probability of a Cat Living to a Certain Age}
\phantomsection\label{example-probability-of-a-cat-living-to-a-certain-age}
\begin{itemize}
\tightlist
\item
  The lifespan of domestic cats is normally distributed with a mean of
  15.7 years and a standard deviation of 1.6 years.
\end{itemize}

\begin{columns}[T]
\begin{column}{0.48\textwidth}
\vspace{1cm}

\begin{itemize}
\tightlist
\item
  \textbf{Question}: What is the probability that a cat will live to be
  as old as Allison's 18-year-old cat?
\end{itemize}
\end{column}

\begin{column}{0.48\textwidth}
\end{column}
\end{columns}
\end{frame}

\begin{frame}{Example: Probability of a Cat Living to a Certain Age}
\phantomsection\label{example-probability-of-a-cat-living-to-a-certain-age-1}
\begin{itemize}
\tightlist
\item
  The lifespan of domestic cats is normally distributed with a mean of
  15.7 years and a standard deviation of 1.6 years.
\end{itemize}

\begin{columns}[T]
\begin{column}{0.48\textwidth}
\vspace{1cm}

\begin{itemize}
\tightlist
\item
  \textbf{Question}: What is the probability that a cat will live to be
  as old as Allison's 18-year-old cat?
\end{itemize}

\begin{itemize}
\tightlist
\item
  We're looking for the probability P(X \textgreater{} 18), which
  represents the probability that a cat will live longer than 18 years.
\end{itemize}
\end{column}

\begin{column}{0.48\textwidth}
\includegraphics{M5-Hypothesis-Testing,-Probability-and-Distribution_files/figure-beamer/unnamed-chunk-27-1.pdf}
\end{column}
\end{columns}
\end{frame}

\begin{frame}{\textbf{Steps 1}:}
\phantomsection\label{steps-1}
\begin{itemize}
\tightlist
\item
  \textbf{Calculate the Z-score}:
\end{itemize}

\begin{columns}[T]
\begin{column}{0.48\textwidth}
\[
   Z = \frac{X - \mu}{\sigma}
   \]

\begin{itemize}
\tightlist
\item
  where:

  \begin{itemize}
  \tightlist
  \item
    \(X\) is the value (18 years),
  \item
    \(\mu\) is the mean (15.7 years),
  \item
    \(\sigma\) is the standard deviation (1.6 years).
  \end{itemize}
\end{itemize}
\end{column}

\begin{column}{0.48\textwidth}
\vspace{1cm}

\[
   Z = \frac{18 - 15.7}{1.6} = \frac{2.3}{1.6} \approx 1.4375
   \]
\end{column}
\end{columns}
\end{frame}

\begin{frame}[fragile]{Step 2.}
\phantomsection\label{step-2.}
\begin{itemize}
\tightlist
\item
  Find the Probability:
\end{itemize}

\begin{Shaded}
\begin{Highlighting}[]
\CommentTok{\# Given values}
\NormalTok{mean\_lifespan }\OtherTok{\textless{}{-}} \FloatTok{15.7}
\NormalTok{sd\_lifespan }\OtherTok{\textless{}{-}} \FloatTok{1.6}
\NormalTok{age\_allison\_cat }\OtherTok{\textless{}{-}} \DecValTok{18}
\CommentTok{\# Calculate Z{-}score}
\NormalTok{z\_score }\OtherTok{\textless{}{-}}\NormalTok{ (age\_allison\_cat }\SpecialCharTok{{-}}\NormalTok{ mean\_lifespan) }\SpecialCharTok{/}\NormalTok{ sd\_lifespan}
\CommentTok{\# Find the probability that a cat lives longer than 18 years}
\NormalTok{probability }\OtherTok{\textless{}{-}} \DecValTok{1} \SpecialCharTok{{-}} \FunctionTok{pnorm}\NormalTok{(z\_score)}
\NormalTok{probability}
\end{Highlighting}
\end{Shaded}

\begin{verbatim}
## [1] 0.07528799
\end{verbatim}

\begin{itemize}
\tightlist
\item
  Thus, the probability that a cat will live to be as old as or older
  than 18 years is approximately \textbf{0.0749} or \textbf{7.49\%}.
\end{itemize}
\end{frame}

\begin{frame}[fragile]{Step 2.}
\phantomsection\label{step-2.-1}
\begin{itemize}
\tightlist
\item
  Find the Probability:
\end{itemize}

\begin{Shaded}
\begin{Highlighting}[]
\CommentTok{\# Given values}
\NormalTok{mean\_lifespan }\OtherTok{\textless{}{-}} \FloatTok{15.7}
\NormalTok{sd\_lifespan }\OtherTok{\textless{}{-}} \FloatTok{1.6}
\NormalTok{age\_allison\_cat }\OtherTok{\textless{}{-}} \DecValTok{18}
\CommentTok{\# Calculate Z{-}score}
\NormalTok{z\_score }\OtherTok{\textless{}{-}}\NormalTok{ (age\_allison\_cat }\SpecialCharTok{{-}}\NormalTok{ mean\_lifespan) }\SpecialCharTok{/}\NormalTok{ sd\_lifespan}
\CommentTok{\# Find the probability that a cat lives longer than 18 years}
\NormalTok{probability }\OtherTok{\textless{}{-}} \DecValTok{1} \SpecialCharTok{{-}} \FunctionTok{pnorm}\NormalTok{(z\_score)}
\NormalTok{probability}
\end{Highlighting}
\end{Shaded}

\begin{verbatim}
## [1] 0.07528799
\end{verbatim}

\begin{itemize}
\tightlist
\item
  This example walks you through calculating the probability in R using
  \texttt{pnorm()}, which calculates the cumulative probability under
  the normal distribution.
\end{itemize}
\end{frame}

\begin{frame}{Another real life problems}
\phantomsection\label{another-real-life-problems}
\end{frame}

\begin{frame}{Another real life problems}
\phantomsection\label{another-real-life-problems-1}
\small

\begin{itemize}
\tightlist
\item
  \textbf{Context}: The EPA is assessing drinking water standards to
  protect public health.\\
\end{itemize}

\begin{columns}[T]
\begin{column}{0.5\textwidth}
\small

\begin{itemize}
\tightlist
\item
  \textbf{Problem}: If the EPA sets the maximum allowable lead
  concentration in drinking water at 1ppm, and the lead concentrations
  in public buildings are normally distributed with a mean of 0.6 ppm
  and a standard deviation of 0.2ppm.

  \begin{itemize}
  \tightlist
  \item
    \textbf{What proportion of public buildings will exceed this
    threshold and require lead remediation?}\\
  \end{itemize}
\end{itemize}

\vspace{1cm}
\end{column}

\begin{column}{0.5\textwidth}
\end{column}
\end{columns}
\end{frame}

\begin{frame}[fragile]{Another real life problems}
\phantomsection\label{another-real-life-problems-2}
\small

\begin{itemize}
\tightlist
\item
  \textbf{Context}: The EPA is assessing drinking water standards to
  protect public health.\\
\end{itemize}

\begin{columns}[T]
\begin{column}{0.5\textwidth}
\small

\begin{itemize}
\tightlist
\item
  Analysis Required:

  \begin{itemize}
  \tightlist
  \item
    Calculate the z-score for the threshold of 1 ppm.
  \item
    Determine the proportion of buildings exceeding this lead level
    using the normal distribution.\\
  \end{itemize}
\end{itemize}
\end{column}

\begin{column}{0.5\textwidth}
\begin{Shaded}
\begin{Highlighting}[]
\CommentTok{\# Parameters}
\NormalTok{mean }\OtherTok{\textless{}{-}} \FloatTok{0.6}
\NormalTok{sd }\OtherTok{\textless{}{-}} \FloatTok{0.2}
\CommentTok{\# Threshold 4 lead remediation}
\NormalTok{threshold }\OtherTok{\textless{}{-}} \DecValTok{1}
\CommentTok{\# Calculate the z{-}score for the threshold}
\NormalTok{z\_score}\OtherTok{\textless{}{-}}\NormalTok{(threshold}\SpecialCharTok{{-}}\NormalTok{mean)}\SpecialCharTok{/}\NormalTok{sd}
\CommentTok{\# Calculate the proportion of }
\CommentTok{\#buildings above the threshold}
\NormalTok{threshold}\OtherTok{\textless{}{-}}\DecValTok{1}\SpecialCharTok{{-}}\FunctionTok{pnorm}\NormalTok{(z\_score)}
\NormalTok{threshold}
\end{Highlighting}
\end{Shaded}

\begin{verbatim}
## [1] 0.02275013
\end{verbatim}
\end{column}
\end{columns}
\end{frame}

\begin{frame}{}
\phantomsection\label{section-1}
\end{frame}

\end{document}
