% Options for packages loaded elsewhere
\PassOptionsToPackage{unicode}{hyperref}
\PassOptionsToPackage{hyphens}{url}
%
\documentclass[
  ignorenonframetext,
]{beamer}
\usepackage{pgfpages}
\setbeamertemplate{caption}[numbered]
\setbeamertemplate{caption label separator}{: }
\setbeamercolor{caption name}{fg=normal text.fg}
\beamertemplatenavigationsymbolsempty
% Prevent slide breaks in the middle of a paragraph
\widowpenalties 1 10000
\raggedbottom
\setbeamertemplate{part page}{
  \centering
  \begin{beamercolorbox}[sep=16pt,center]{part title}
    \usebeamerfont{part title}\insertpart\par
  \end{beamercolorbox}
}
\setbeamertemplate{section page}{
  \centering
  \begin{beamercolorbox}[sep=12pt,center]{part title}
    \usebeamerfont{section title}\insertsection\par
  \end{beamercolorbox}
}
\setbeamertemplate{subsection page}{
  \centering
  \begin{beamercolorbox}[sep=8pt,center]{part title}
    \usebeamerfont{subsection title}\insertsubsection\par
  \end{beamercolorbox}
}
\AtBeginPart{
  \frame{\partpage}
}
\AtBeginSection{
  \ifbibliography
  \else
    \frame{\sectionpage}
  \fi
}
\AtBeginSubsection{
  \frame{\subsectionpage}
}
\usepackage{amsmath,amssymb}
\usepackage{iftex}
\ifPDFTeX
  \usepackage[T1]{fontenc}
  \usepackage[utf8]{inputenc}
  \usepackage{textcomp} % provide euro and other symbols
\else % if luatex or xetex
  \usepackage{unicode-math} % this also loads fontspec
  \defaultfontfeatures{Scale=MatchLowercase}
  \defaultfontfeatures[\rmfamily]{Ligatures=TeX,Scale=1}
\fi
\usepackage{lmodern}
\usetheme[]{Boadilla}
\ifPDFTeX\else
  % xetex/luatex font selection
\fi
% Use upquote if available, for straight quotes in verbatim environments
\IfFileExists{upquote.sty}{\usepackage{upquote}}{}
\IfFileExists{microtype.sty}{% use microtype if available
  \usepackage[]{microtype}
  \UseMicrotypeSet[protrusion]{basicmath} % disable protrusion for tt fonts
}{}
\makeatletter
\@ifundefined{KOMAClassName}{% if non-KOMA class
  \IfFileExists{parskip.sty}{%
    \usepackage{parskip}
  }{% else
    \setlength{\parindent}{0pt}
    \setlength{\parskip}{6pt plus 2pt minus 1pt}}
}{% if KOMA class
  \KOMAoptions{parskip=half}}
\makeatother
\usepackage{xcolor}
\newif\ifbibliography
\usepackage{longtable,booktabs,array}
\usepackage{calc} % for calculating minipage widths
\usepackage{caption}
% Make caption package work with longtable
\makeatletter
\def\fnum@table{\tablename~\thetable}
\makeatother
\usepackage{graphicx}
\makeatletter
\def\maxwidth{\ifdim\Gin@nat@width>\linewidth\linewidth\else\Gin@nat@width\fi}
\def\maxheight{\ifdim\Gin@nat@height>\textheight\textheight\else\Gin@nat@height\fi}
\makeatother
% Scale images if necessary, so that they will not overflow the page
% margins by default, and it is still possible to overwrite the defaults
% using explicit options in \includegraphics[width, height, ...]{}
\setkeys{Gin}{width=\maxwidth,height=\maxheight,keepaspectratio}
% Set default figure placement to htbp
\makeatletter
\def\fps@figure{htbp}
\makeatother
\setlength{\emergencystretch}{3em} % prevent overfull lines
\providecommand{\tightlist}{%
  \setlength{\itemsep}{0pt}\setlength{\parskip}{0pt}}
\setcounter{secnumdepth}{-\maxdimen} % remove section numbering
\usepackage{graphicx}
\logo{\ifnum\thepage>1\hfill\includegraphics[width=1cm]{logo}\fi}
\titlegraphic{\includegraphics[width=3cm]{logo}}
\newcommand{\theHtable}{\thetable}
\ifLuaTeX
  \usepackage{selnolig}  % disable illegal ligatures
\fi
\usepackage{bookmark}
\IfFileExists{xurl.sty}{\usepackage{xurl}}{} % add URL line breaks if available
\urlstyle{same}
\hypersetup{
  pdftitle={Hypothesis Testing, Probability and Distributions},
  pdfauthor={Pablo E. Gutiérrez-Fonseca},
  hidelinks,
  pdfcreator={LaTeX via pandoc}}

\title{Hypothesis Testing, Probability and Distributions}
\author{Pablo E. Gutiérrez-Fonseca}
\date{Fall 2024}

\begin{document}
\frame{\titlepage}

\begin{frame}{Normality, Probability and Significance}
\phantomsection\label{normality-probability-and-significance}
\begin{itemize}
\tightlist
\item
  Why did we focus on normality?

  \begin{itemize}
  \tightlist
  \item
    The \textbf{normal distribution} is a key tool for determining the
    probability of a given value occurring in a population that follows
    this distribution.
  \end{itemize}
\end{itemize}

\begin{itemize}
\tightlist
\item
  It allows us to make inferences about a population by calculating how
  likely it is for data to fall within certain ranges.
\end{itemize}

\begin{itemize}
\tightlist
\item
  Many statistical tests assume data follows a \textbf{normal
  distribution}, which helps in determining \textbf{significance} and
  making reliable conclusions.
\end{itemize}
\end{frame}

\begin{frame}{Hypothesis Testing}
\phantomsection\label{hypothesis-testing}
\begin{itemize}
\tightlist
\item
  All inferential tests use a formula that calculates a \textbf{test
  statistic}, quantifying the relationship or difference you are
  testing.
\end{itemize}

\begin{columns}[T]
\begin{column}{0.48\textwidth}
\vspace{1cm}

\begin{itemize}
\item
  Independent t-test:
  \[t = \frac{\bar{X_1} - \bar{X_2}}{\sqrt{\frac{s_1^2}{n_1} + \frac{s_2^2}{n_2}}}\]

  \begin{itemize}
  \item
    Dependent t-test \[t = \frac{\bar{D}}{\frac{s_D}{\sqrt{n}}}\]
  \item
    One sample z-test
    \[z = \frac{\bar{X} - \mu}{\frac{\sigma}{\sqrt{n}}}\]
  \end{itemize}
\end{itemize}
\end{column}

\begin{column}{0.48\textwidth}
\vspace{1cm}

\begin{itemize}
\item
  F-test (ANOVA) \[F = \frac{MST}{MSE}\]
\item
  Pearson correlation
  \[r = \frac{\sum (x_i - \bar{x})(y_i - \bar{y})}{\sqrt{\sum (x_i - \bar{x})^2 \sum (y_i - \bar{y})^2}}\]
\end{itemize}
\end{column}
\end{columns}
\end{frame}

\end{document}
